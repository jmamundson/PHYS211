\section{Introduction to waves}
Objectives:
\begin{itemize}
\item Types of waves
\item Wave model
\end{itemize}

\hrulefill

\subsection{Types of waves}
Waves are a type of oscillation. There are many types of waves:
\begin{itemize}
\item sound waves -- gas
\item elastic waves -- solid
\item water waves -- liquid
\item electromagnetic waves -- don't require a medium...
\end{itemize}

These can be categorized as two different types of waves.\\
(1) Mechanical waves:
\begin{itemize}
\item Motion of a substance caused by a disturbance to the substance. Mechanical waves include sound, elastic, and water waves.
\item Wave speed, $v$, depends on material properties.
\item No net motion of material. Waves transmit energy, not particles. (Demonstrate with a slinky.)
\end{itemize}

(2) Electromagnetic waves:
\begin{itemize}
\item Waves of an electromagnetic field. Includes visible light, radio waves, x-rays, ...
\item Not due to the motion of a substance. EM waves can travel through vacuums!
\end{itemize}

The details of wave behavior differ, but we can discuss basic properties of waves with a ``wave model''.

The most basic waves are transverse (wave on a string) or longitudinal (sound wave). For transverse waves, the particle motion is perpendicular to the wave direction, whereas for longitudinal waves the particle motion is parallel to the wave direction.

Waves can travel as a single pulse or a series of pulses. When waves are produced by a simple harmonic oscillator, the waves are sinusoidal.

\subsection{One-dimensional waves}
We will focus on describing waves that travel in one direction. Waves are a function of two variables: position and time. We need a function that describes when and where a wave is located. We'll start graphically, by considering simple wave pulses.
\clearpage
Snapshot graphs show what a wave looks like at any instant in time.
\vspace{8cm}

History graphs show the history of a particular point in the medium.
\vspace{6cm}
\clearpage
\subsubsection{Example \#1: Sketch a history graph}
\vspace{8cm}

\subsubsection{Example \#2: Sketch a snapshot graph}
\vspace{8cm}

\clearpage
\subsection{Wave model}
[Insert diagram of a sinusoidal wave; show how its position changes at some later time.]
\vspace{5cm}

We can describe the wave displacement, $y$, by

$$y(x,t)=A\sin\left(2\pi\frac{x}{\lambda}\pm 2\pi\frac{t}{T}\right)$$

The displacement depends on both location and time. $x/\lambda$ describes the wave shape. The $+$ indicates that the wave is travelling to the left, the $-$ indicates that it is travelling to the right. Note that the wave model uses three dimensions ($x$, $y$, and $t$)... Also note that this is written in terms of a transverse wave, but it also applies to longitudinal waves. The vertical displacement can just be replaced with a horizontal displacement relative to an object's initial position.

Let's analyze this equation in a bit of detail. 

\subsubsection{Oscillations of fixed points}
First, consider fixed points in space, e.g., $x=0$. Then
$$y(0,t)=A\sin\left(\pm 2\pi\frac{t}{T}\right)$$

Each particle in the waves oscillates sinusoidally with amplitude $A$.

What if $x=\lambda/2$?
$$y\left(\frac{\lambda}{2},t\right)=A\sin\left(2\pi\frac{\lambda/2}{\lambda}\pm 2\pi\frac{t}{T}\right)=A\sin\left(\pi\pm 2\pi\frac{t}{T}\right)$$
The $\pi$ at the beginning is just a ``phase'' shift. To see how, let's recall an identity from trigonometry:
$$\sin(\alpha\pm\beta)=\sin\alpha\cos\beta\pm\cos\alpha\sin\beta$$
Here, $\alpha=\pi$; $\sin\pi=-1$ and $\sin\pi=0$. This means that
$$y\left(\frac{\lambda}{2},t\right)=-A\sin\left(\pm 2\pi\frac{t}{T}\right)$$
which is what we would expect. If we looked at $x=\lambda$, we would find that 
$$y(\lambda,t)=A\sin\left(\pm 2\pi\frac{t}{T}\right)$$
which is the same as for $x=0$. This is good!

\subsubsection{Snapshots in time}
Now let's take a look at snapshots in time for a wave that is travelling to the right (use negative sign in the equation). If $t=0$, this means that
$$y(x,0)=A\sin\left(2\pi\frac{x}{\lambda}\right)$$
When $t=T/4$, this gives
$$y\left(x,\frac{T}{4}\right)=A\sin\left(2\pi\frac{x}{\lambda}-\frac{\pi}{2}\right),$$
which represents a 90$^\circ$ phase shift to the right. To see this, we can again use the angle sum and difference identity. In this case, $\beta=\pi/2$, so $\sin\beta=0$ and $\sin\beta=1$. This means that
$$y\left(x,\frac{T}{4}\right)=A\sin\left(2\pi\frac{x}{\lambda}\right)$$

[Insert diagram.]
\vspace{5cm}

\subsubsection{Wave speed}
The wave model characterizes the speed at which a wave propagates to the left or right.
$$v=\frac{\Delta{x}}{\Delta{t}}=\frac{\lambda}{T}\Rightarrow \lambda=vT$$
Plugging this into the wave model gives
$$y(x,t)=A\sin\left(2\pi\frac{x}{vT}\pm 2\pi\frac{t}{T}\right)=A\sin\left(2\pi\frac{1}{T}\left(\frac{x}{v}\pm t\right)\right)=A\sin\left(2\pi f\left(\frac{x}{v}\pm t\right)\right)$$

\clearpage
\subsubsection{Examples: Find wave properties}
For the waves shown below, find the (i) wavelength, (ii) period, (iii) wave speed, and (iv) phase constant.

\clearpage
