\section{WORK AND ENERGY, PART 2}
Objectives:
\begin{itemize}
\item Thermal energy from friction
\item Thermal energy from collisions
  \end{itemize}

\hrulefill

Last class:
\begin{itemize}
\item Work: $W=Fd=\Delta{E}=\Delta{K}+\Delta{U_g}+\Delta{E_{th}}+...$
\item Translational kinetic energy: $K_{trans}=\frac{1}{2}mv^2$
\item Rotational kinetic energy: $K_{rot}=\frac{1}{2}I\omega^2$
\item Gravitational potential energy: $\Delta{U_g}=mg\Delta{y}$; always from some reference state
\end{itemize}

\subsection{Demo with basketball and tennis ball}
First analyze collision, using reference frame that moves with the basketball.
$$m_1v_{1,i}=m_1v_{1,f}+m_2v_{2,f}$$
$$\frac{1}{2}m_1v_{1,i}^2=\frac{1}{2}m_1v_{1,f}^2+\frac{1}{2}m_2v_{2,f}^2$$
From momentum equation,
$$v_{1,f}=v_{1,i}-\frac{m_2}{m_1}v_{2,f}$$
Plug into energy equation,
$$\frac{1}{2}m_1\left(v_{1,i}-\frac{m_2}{m_1}v_{2,f}\right)^2=\frac{1}{2}m_1v_{1,f}^2+\frac{1}{2}m_2v_{2,f}^2$$
With some algebra and using the quadratic equation...
$$\boxed{v_{2,f}=\frac{2m_1}{m_1+m_2}v_{1,i}}$$
and
$$\boxed{v_{1,f}=\frac{m_1-m_2}{m_1+m_2}v_{1,i}}$$

Okay, now, how high will the tennis ball bounce (using conservation of energy)? Both balls fall and gain kinetic energy.
$$mg\Delta y + \frac{1}{2}m\left(v_1^2-v_0^2\right)$$
$$v_1 = \sqrt{-2g\Delta y} = \sqrt{2gH}$$

If basketball has the same speed after bouncing off the floor, then its velocity is $\sqrt{2gH}$ and the speed of the tennis ball is $-\sqrt{2gH}$.

Use collision equations to see what happens to the tennis ball. In the reference frame that is moving upward at $\sqrt{2gH}$, we can use the equations above for the collision. We want to solve for $v_{1,f}$, but now note that $v_{1,i}=-2\sqrt{2gH}$.
$$v_{1,f}=\frac{m_1-m_2}{m_1+m_2}v_{1,i}=-\frac{m_1-m_2}{m_1+m_2}2\sqrt{2gH}$$
That is result in moving reference frame. In stationary reference frame, add $\sqrt{2gH}$

So right after the collision the speed is
$$v_1 = -\frac{m_1-m_2}{m_1+m_2}2\sqrt{2gH}+\sqrt{2gH}=\sqrt{2gH}\left(1-\frac{2(m_1-m_2)}{m_1+m_2}\right)$$

weight of basketball = 0.6 kg; weight of tennis ball = 0.05 kg

plug values in, get
$$v_1 \approx = 2.7\sqrt{2gH}$$

height that it bounces
$$\Delta y = \frac{v_i^2}{2g}=...$$


\subsection{Collisions}
We've talked about how thermal energy can be generated by friction. It can also be generated by collisions. There are two types of collisions: elastic and inelastic. 

In \textit{perfectly} elastic collisions, the objects bounce off of each other and the kinetic energy of the system is conserved. In inelastic collisions, the objects stick together and the kinetic energy is \textit{not} conserved.

\subsubsection{Inelastic collisions}
Consider two objects that are involved in an inelastic collision. One object is moving and the other is stationary. Conservation of momentum tells us that
$$m_1v_i=(m_1+m_2)v_f$$
The change in kinetic energy during this collision is
$$\Delta{K}=\frac{1}{2}(m_1+m_2)v_f^2-\frac{1}{2}m_1v_i^2$$
Solving the momentum equation for $v_f$, plugging the result into $\Delta{K}$, and re-arranging, we see that
$$\Delta{K}=-K_i\left(\frac{m_2}{m_1+m_2}\right)=\Delta{E_{th}}$$
\begin{itemize}
\item If $m_2\gg m_1$, then most of the energy goes into thermal energy. 
\item If $m_1\gg m_2$, then very little thermal energy is created. 
\item If $m_1\approx m_2$, then half of the energy is transformed into thermal energy.
\end{itemize}

\subsubsection{Elastic collisions}
Elastic collisions are often described using a coefficient of restitution.
$$v_f=C_rv_i$$
where $C_r$ is empirically determined and $0\leq C_r \leq 1$. The coefficient of restitution is actually described as a ratio of relative velocities, but to simplify things we'll assume that one of the objects is immovable (a wall, the floor, etc.).

From this, we can ask how much heat is generated during a collision?

Consider a cart moving on a frictionless track that collides with a wall. Define the cart and the track/ground as being the system, so that there are no external forces.

$$0=\Delta{E}=\Delta{K}+\Delta{E_{th}}$$
Here, $\Delta{E_{th}}$ refers to the thermal energy generated by the collision. So,
$$0=\frac{1}{2}mv_f\ds^2-\frac{1}{2}mv_i\ds^2+\Delta{E_{th}}$$
$$0=\frac{1}{2}m(C_rv_i)\ds^2-\frac{1}{2}mv_i\ds^2+\Delta{E_{th}}$$
$$0=\frac{1}{2}mC_r\ds^2v_i\ds^2-\frac{1}{2}mv_i\ds^2+\Delta{E_{th}}$$
$$0=\frac{1}{2}mv_i\ds^2(C_r\ds^2-1)+\Delta{E_{th}}$$
$$\Delta{E_{th}}=\frac{1}{2}mv_i\ds^2(1-C_r\ds^2)=K_i(1-C_r\ds^2)$$

\subsubsection{Example \#1: Ball bounces off the floor}
A ball is dropped from a height of 1 m. The coefficient of restitution between the ball and the floor is 0.8. How high does the ball bounce?

$$\Delta{E}=0=\Delta{K}+\Delta{U_g}+\Delta{E_{th}}$$
Initially and at the peak, $K=0$ so $\Delta{K}=0$. So
$$0=\Delta{U_g}+\Delta{E_{th}}=mg\Delta{y}+K_b(1-C_r\ds^2),$$
where $K_b$ is the kinetic energy right before the collision. If we take the ground as our reference state, then $K_b=U_i=mgy_i$ (all of the energy goes into kinetic energy).  Therefore,
$$0=mg\Delta{y}+mgy_i(1-C_r\ds^2)=y_f-y_i+y_i-y_iC_r\ds^2=y_f-y_iC_r\ds^2$$
And so
$$y_f=y_iC_r\ds^2=1\mbox{ m}\cdot(0.8)^2=0.64\mbox{ m}$$

(Notice that this equation here is one simple way to figure out an object's coefficient of restitution.)


\subsection{Thermal Energy}
Let's add a bit of complexity by thinking about friction and thermal energy. We'll start with a simple system: a box is pulled across a horizontal, rough surface. What is the work done on the box by friction?

$$\sum F_x=F_t=ma_x$$
The only external force acting on the system is the tension from the rope, which has to equal the frictional force between the box and the floor (otherwise the box wouldn't move with constant speed). Defining the system as the box and the floor, the work done on the box is:
$$W=\ds\int_{x_i}^{x_f}F_t\,dx=F_t\Delta{x}=F_k\Delta{x}=\Delta{E}$$
There was a change in energy, but the box did not (i) speed up or (ii) change elevation. This work caused molecules at the box's surface to vibrate (i.e., heat up), so we refer to this as thermal energy.
$$\boxed{\Delta{E_{th}}=F_k\Delta{x}}$$

\subsubsection{Example \#1: You sled down a hill}
You sled down a 3-m tall hill. The hill is frictionless, but there is some soft snow at the bottom that has a coefficient of kinetic friction of $\mu_k=0.05$. The ground at the bottom of the hill is horizontal. How far will you slide before coming to rest?

[Insert diagram of hill.]
\vspace{5cm}

How would we have we solved this previously? Its fairly involved, but we could do this using forces and kinematics. But its super easy using conservation of energy.

$$\Delta{E}=\Delta{K}+\Delta{U_g}+\Delta{E_{th}}=0$$.
You start at rest and finish at rest, so $\Delta{K}=$. This equation therefore becomes
$$mg\Delta{y}+F_k\Delta{x}=0,$$
where $\Delta{x}$ is the distance that you travel after reaching the bottom of the hill. Recall that the frictional force is given by
$$F_k=\mu_kF_n$$
In this problem, it easy to see that $F_n=F_g=mg$ when you are sliding on horizontal ground. So, this means that
$$mg\Delta{y}+\mu_k mg\Delta{x}=0$$
$$\Delta{y}+\mu_k \Delta{x}=0\Rightarrow \boxed{\Delta{x}=\frac{-\Delta{y}}{\mu_k}=60\mbox{ m}}$$
Note that the solution does not depend on the angle or length of the hill!


\clearpage
