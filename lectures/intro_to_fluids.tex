\section{Introduction to fluids}
Objectives:
\begin{itemize}
\item Review thermodynamics of gases
\item Hydrostatics
\end{itemize}

\hrulefill

\subsection{Thermodynamics of gases}
Important ideas from last time:
\begin{itemize}
\item Ideal gas law: $PV=nRT$. For closed containers, we can saw that $\frac{PV}{T}=\mbox{constant}$.
\item Change in thermal energy of a \textit{monatomic} gas is given by $\Delta{E_{th}}=\frac{3}{2}nR\Delta{T}$
\item Work and heat (transfer) can change the total energy of a gas
\item Isochoric or isovolumetric process: heat transferred into a gas container with no change in volume
  $$Q=\frac{3}{2}nR\Delta{T}=nC_v\Delta{T}$$
\item Isobaric process: pressure is constant, but volume and temperature can change due to work or heat transfer
  $$Q=\frac{5}{2}nR\Delta{T}=nC_p\Delta{T}$$
\item Isothermal process: no change in temperature (or thermal energy), so work done by a gas must balance with heat transfer into the gas
  $$W_{gas}=Q$$
\item Adiabatic process: no heat transfer between the system and the environment
  $$W=\Delta{E_{th}}$$
\item Work done to/by a gas is the area under the pressure-volume curve. A consequence of this is that the heat transfer into or out of a gas depends on the \textit{path}.  
\end{itemize}

[Insert PV-diagram showing how different paths result in different amounts of work and heat.]
\vspace{6cm}

\subsubsection{Example \#1: Expandable cube}
An expandable cube, initially 20 cm on each side, contains 3.0 g of helium at 20$^\circ$C. 1000 J of heat are transferred into this gas. What are (a) the final pressure if the process is at a constant volume and (b) the final volume if the process is at constant pressure?

Also given:\\
3 g of He is 0.75 mol\\
$C_v=12.5$ J/(mol$\cdot$K)\\
$C_p=20.8$ J/(mol$\cdot$K)

(a) Since we want to find $P_f$, let's rearrange the ideal gas law:
$$P_f = \frac{nRT_f}{V}$$
What is $T_f$? We know that $T_i=293\mbox{ K}$. Heating the gas will change its temperature
$$Q=nC_v\Delta{T}\Rightarrow \Delta{T}=\frac{Q}{nC_v}=107\mbox{ K},$$
so $T_f=400$ K.

Plugging this into the equation for $P_f$ gives
$$\boxed{P_f=311\mbox{ kPa}}$$

(b) Because this is constant pressure, we have
$$\frac{PV}{T}=\mbox{constant}\Rightarrow \frac{V}{T}=\mbox{constant}$$
This means that
$$\frac{V_i}{T_i}=\frac{V_f}{T_f}\Rightarrow V_f=\frac{T_f}{T_i}V_i$$

The initial volume and temperature are given. The initial volume is $(0.20\mbox{ m})^3=0.008\mbox{ m}^3=8\mbox{ L}$. We just need to calculate $T_f$.
$$Q=nC_p\Delta{T}\Rightarrow \Delta{T}=\frac{Q}{nC_p}=64.1\mbox{ K},$$
so $T_f=357\mbox{ K}$. Consequently, 
$$\boxed{V_f=0.00975\mbox{ m}^3=9.75\mbox{ L}}$$.

\subsection{Hydrostatics}
That's all that I have to say (for now, anyway) about thermodynamics and gases. I'd like to now turn our attention to fluids. We'll first focus on hydrostatics --- fluids at rest.

What is a fluid? Its a substance that flows and takes the shape of a container.

\clearpage
[Diagram of container containing a fluid.]
\vspace{5cm}

In a fluid, molecules are weakly bonded but can slide past each other.

An important parameter for describing fluids is density.
$$\rho=\frac{m}{V}$$


Examples:\\
sea water: $\sim 1030$ kg/m$^3$; it depends on salinity, temperature, and pressure

fresh water: $\sim 1000$ kg/m$^3$; it depends on temperature\\
\begin{itemize}
\item at 20$^\circ$C, $\rho=998.2071$ kg/m$^3$
\item at 4$^\circ$C, $\rho=999.9720$ kg/m$^3$
\item at 0$^\circ$C, $\rho=998.8395$ kg/m$^3$
\end{itemize} 

ice: $\sim 917$ kg/m$^3$ (if bubble free, otherwise $<917$ kg/m$^3$)

firn: $\sim 800$ kg/m$^3$

fresh snow: $100$--$300$ kg/m$^3$

Water is \underline{weird}. For most substances, the solid form is more dense than the liquid form. If this were also true for water, life wouldn't exist --- at least not as we know it. Its also weird that the densest fresh water is 4$^\circ$C.

Ice that forms in a water body floats at the surface, and it insulates the water from cold air. The water at depth remains liquid.

[Draw diagram.]
\vspace{5cm}

Its very difficult to form thick lake ice or sea ice. ``Multi-year'' ice in the Arctic Ocean is just a few meters thick.

One important process involving water is the overturning of lakes during spring and fall, which is due to the fact that the densest water is at 4$^\circ$C.

[Draw diagrams showing overturning lakes and temperature profiles.]
\vspace{5cm}

%\hrulefill\\
%\clearpage

\clearpage
