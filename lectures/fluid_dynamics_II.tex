\section{Fluid dynamics; viscosity}
Objectives:
\begin{itemize}
\item Venturi tube
\item Viscous flow
\end{itemize}

\hrulefill

Recall key ideas of fluid dynamics:
\begin{enumerate}
\item Ideal fluid: incompressible, laminar, inviscid
\item Flux continuity: $Q=vA=\mbox{constant}$
\item Bernoulli's equation: $P+\frac{1}{2}\rho v^2+\rho gh=\mbox{constant}$
\end{enumerate}

\subsection{Venturi tube}
One really useful application of Bernoulli's Equation is that it can be used to measure fluid flow through pipes (using a Venturi tube).

[Insert diagram.]
\vspace{5cm}

If we look at the streamline that passes through the center of this tube,
$$P_1+\frac{1}{2}\rho v_1\ds^2+\rho gy_1=P_2+\frac{1}{2}\rho v_2\ds^2+\rho gy_2$$
But points 1 and 2 are at the same elevation, so $y_1=y_2$, and so
$$P_1+\frac{1}{2}\rho v_1\ds^2=P_2+\frac{1}{2}\rho v_2\ds^2$$
We can re-arrange this so that
$$P_2-P_1=\frac{1}{2}\rho(v_1\ds^2-v_2\ds^2)$$

From flux continuity, 
$$v_1A_1=v_2A_2$$
so 
$$v_1=v_2\left(\frac{A_2}{A_1}\right)$$

Plugging this into Bernoulli's equation gives
$$P_2-P_1=\frac{1}{2}\rho\left(\left(\frac{A_2}{A_1}\right)^2v_2\ds^2-v_2^2\right)=\frac{1}{2}\rho v_2\ds^2\left(\left(\frac{A_2}{A_1}\right)^2-1\right)$$

What is $P_2-P_1$? Because there is no fluid flow in the vertical direction, it must be that the pressures $P_1$ and $P_2$ must equal the hydrostatic pressure of the overlying columns of water. In other words,
$$P_2 = P_{atm}+\rho g h_2$$
and
$$P_1 = P_{atm}+\rho g h_1.$$
Combining these things,
$$P_2-P_1=\rho g (h_2-h_1) = \rho gd$$
Inserting this into Bernoulli's equation,
$$\rho gd=\frac{1}{2}\rho v_2\ds^2\left(\left(\frac{A_2}{A_1}\right)^2-1\right)$$

Solving for $v_2$ gives
$$\boxed{v_2=\sqrt{\frac{2gd}{\left(\left(\frac{A_2}{A_1}\right)^2-1\right)}}=A_1\sqrt{\frac{2gd}{\left(A_2^2-A_1^2\right)}}=0.36\mbox{ m/s}}$$

And this means that
$$\boxed{v_1=1.45\mbox{ m/s}}$$

\subsection{Viscous flow}
Let's see what happens if we relax the condition that the fluid is inviscid.

[Insert diagram of fluid flow through a pipe with friction.]
\vspace{5cm}

From conservation of energy:
$$W=\Delta{K}+\Delta{U_g}+\Delta{E_{th}}$$
Following the same steps as when we derived Bernoulli's equation,
$$(P_1-P_2)V=\frac{1}{2}\rho V(v_2\ds^2-v_1\ds^2)+\rho gV(y_2-y_1)+\Delta{E_{th}}$$
If we consider a horizontal pipe with uniform diameter, $y_1=y_2$ and flux continuity requires that $v_1=v_2$. Therefore,
$$\Delta{E_{th}}=(P_1-P_2)V$$

Friction within the fluid, and between the fluid and walls of the pipe, cause $\Delta E_{th}>0$. This means that $P_2<P_1$.

Experiments show that the pressure difference need to drive a viscous fluid through a pipe is
$$P_1-P_2=8\pi\eta\frac{Lv_{\rm avg}}{A}$$
where $\eta$ is the fluid viscosity and has units of Pa$\cdot$s. The viscosity of water is about 10$^{-3}$ Pa$\cdot$s, while the viscosity of honey is 20--600 Pa$\cdot$s. This equation is referred to as Poiseuille's Equation. It tells us two things. First, the average velocity of fluid flowing through a section of pipe is 
$$v_{avg}=\frac{(P_1-P_2)}{L}\frac{A}{8\pi\eta}$$
In other words, you need a pressure gradient to drive flow, large pipes provide less resistance, and fluids with high viscosity flow slowly. How do you keep the velocity high?

Examples:
\begin{enumerate}
\item Alaska pipeline: Use pump stations to create large pressure gradients.
\item Clogged arteries: Use blood thinner to reduce $\eta$, or scraping out arteries to increase $A$.
\end{enumerate}

Second, we can relate this to the amount of thermal energy that is produced in a section of fluid.
$$\Delta E_{th}=8\pi\eta\frac{Lv_{avg}}{A}{V}$$
where $V=A\Delta x$, so
$$\Delta E_{th}=8\pi\eta Lv_{avg}\Delta{x}$$
The amount of thermal energy that is generated in the fluid is proportional to the viscosity, length of pipe, velocity, and displacement of the fluid.

So I've now just opened a can of worms... Viscosity is generally highly temperature dependent, and density is sometimes temperature dependent (so assumption of incompressibility is questionable).

\subsubsection{Example: viscous flow through a tube}
A stiff, 10-cm long tube with an inner diameter of 3.0 mm is attached to a small hole in the side of a tall beaker. The tube sticks out horizontally and is open at one end. The beaker is filled with with 20$^\circ$C water ($\eta=10^{-3}\mbox{ Pa}\cdot\mbox{s}$) to a level 45 cm above the hole, and is continually topped off to maintain that level. What is the volume flow rate (flux) through the tube?

The pressure difference between the ends of the tube is
$$P_1-P_2=(P_{atm}+\rho gh)-(P_{atm})=\rho gh$$

Poiseuille's equation tells us how the pressure difference relates to the flow of the water.
$$P_1-P_2=8\pi\eta\frac{v_{\rm avg}L}{A}=\rho gh$$
So 
$$v_{\rm avg}=\frac{\rho ghA}{8\pi\eta L}$$
And so
$$Q=\frac{\rho ghA^2}{8\pi\eta L}=\frac{\rho gh\pi^2r^4}{8\pi\eta L}=8.8\times 10^{-5}\mbox{ m}^3/\mbox{s}$$



\clearpage
