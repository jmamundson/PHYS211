\section{Types of forces, part II}
Objectives:
\begin{itemize}
\item Frictional forces: static vs. kinematic
\end{itemize}

Last class:
\begin{itemize}
\item Gravitational force: $F_g=\frac{Gm_1m_2}{r^2}$, but use $F_g=mg$ near the Earth's surface.
\item Normal force: $F_n>0$, points perpendicular to surface.
\item Tensional force: $F_t$, points in direction of string. Massless string approximation allows us to assume that the tensional force is constant through the string.
\end{itemize}

\subsection{Frictional force}
The normal force is very important when dealing with friction, which is known through experiments.

Consider the case in which the block is resting on a ramp. This means that friction balances whatever forces would tend to cause acceleration (in this case, down the ramp). 

[Insert diagram of boxing resting on a ramp.]
\vspace{5cm}

In our example, this would mean that
$$\sum F_x=F_g\sin\theta-F_s=ma_x$$
where $F_s$ refers to static friction. Summing the forces in the $y-$direction,
$$\sum F_y=F_n-F_g\cos\theta=0\Rightarrow F_n=F_g\cos\theta \Rightarrow F_g=\frac{F_n}{\cos\theta}$$
Plugging this into the $x-$equation,
$$F_n\frac{\sin\theta}{\cos\theta}-F_s=F_n\tan\theta-F_s=ma_x$$

The frictional force (for static friction) will adjust its magnitude to hold the block in place until a certain threshold is reached will be in static equilibrium until a certain threshold is exceeded. For small angles, $F_s=F_n\tan\theta$ and therefore $a_x=0$. When the angle becomes too large, the box starts to slide. Thus, the maximum force of static friction is
$$\boxed{\max F_s=\mu_s F_n}$$
where $\mu_s=\tan\left(\theta_{max}\right)$ is the coefficient of static friction. It is a system property.

What if the box starts sliding?

[Insert diagram.]
\vspace{5cm}

We can solve for the frictional force.

Summing the forces in the $y$-direction:
$$\sum F_y = F_n-F_g\cos\theta=ma_y=0$$
$$F_n=F_g\cos\theta$$

Summing the forces in the $x$-direction:
$$\sum F_x=F_g\sin\theta-F_k=ma_x\neq 0$$
$$F_k=m(a_x-g\sin\theta)$$

This means that we can measure $F_n$ and $F_k$ experimentally. If we do lots of experiments, we find that $F_k$ is proportional to $F_n$.

[Insert diagram of $F_k$ vs. $F_n$; plots as straight line with $y-$intercept of 0.]
\vspace{5cm}

This means that
$$\boxed{ F_k = \mu_k F_n}$$
where $\mu_k$ is the coefficient of kinetic friction. It is a constant for a given system (combination of surfaces) and is determined experimentally. This is just a model for friction; its actually quite a bit more complicated and is still an active area of research. People that study earthquakes worry about friction a lot!

What makes something have a high coefficient of friction?

Coefficients of friction are generally less than 1, but don't have to be.

If you want to solve a problem in which a box isn't sliding initially, do the following:
\begin{enumerate}
\item Calculate the force that will tend to accelerate the box.
\item Does this force exceed $\max F_s$? If no, then immediately have the frictional force $F_s$. If yes, then the box will start sliding and you will have frictional force $F_k=\mu_k F_n$.
\end{enumerate}
In general, $\mu_s>\mu_k$. In other words, its easier to keep an object moving than it is to get it moving in the first place.

\subsection{Example problems}
\subsubsection{Example \#1}
You pull a box with a rope across a horizontal surface. The box is initially at rest, but you are able to overcome the maximum static friction. Plot the frictional force as a function of time. Let $m=100\mbox{ kg}$, $\mu_s=0.3$, and $\mu_k=0.2$.

[Insert diagram.]
\vspace{5cm}

Before and after the box starts moving,
$$\sum F_y = F_n-F_g=0\Rightarrow F_n=F_g=mg$$

Before the box starts moving,
$$\sum F_x=F_t-F_s=0\Rightarrow F_s=F_t$$

Let's assume that we linearly increase $F_t$ until the box starts moving. This occurs once $F_t\geq \max F_s=\mu_s F_n=\mu_s mg=0.3\times 100\mbox{ kg}\times 9.81\mbox{ m/s}^2\approx 300\mbox{ N}$. Once the box starts moving, we switch to kinetic friction: $F_k=\mu_k mg=0.2\times 100\mbox{ kg}\times 9.81\mbox{ m/s}^2\approx 200\mbox{ N}$.

\clearpage
[Insert plot of results.]
\vspace{5cm}

\subsubsection{Example \#2}
Demo to calculate the coefficient of friction of a block sliding over the table.

What do we need?

Forces on sliding block:
$$\sum F_x=F_t-F_k=m_1a_x$$
$$F_k=F_t-m_1a_x$$

$$\sum F_y=F_n-F_g=0\Rightarrow F_n=F_g$$

Forces on falling block:
$$\sum F_y=F_t-F_g=m_2a_y$$
$$F_t=m_2a_y+m_2g$$

$F_t$ on the falling block is the same as $F_t$ on the sliding block. We have an additional constraint, which is that $a_x=-a_y=a$.

So this means that
$$F_k=m_2a_y+m_2g-m_1a_x=m_2g-(m_1+m_2)a=\mu_kF_n=\mu_km_1g$$
$$\mu_k=\frac{m_2g-(m_1+m_2)a}{m_1g}$$

We need to measure the acceleration.
$$\Delta{x}=v_i\Delta{t}+\frac{1}{2}a\Delta{t}^2\Rightarrow a=\frac{2\Delta{x}}{\Delta{t}^2}$$

We can just as easily measure the vertical displacement.



\clearpage
