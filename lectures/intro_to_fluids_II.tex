\section{Introduction to fluids}
Objectives:
\begin{itemize}
\item Density and pressure
\item Archimedes' principle
\item Measuring gas pressure
\end{itemize}

\hrulefill

\subsection{Density and pressure}
We'll spend the next several lectures discussing the motion of fluids by using the ideas of forces and energy.

A \textit{fluid} is simply a substance that flows, and includes both gases and liquids.

In a \textit{gas}, the molecules move freely with few interactions.

In a \textit{liquid}, the molecules are weakly bound to each other but can slide past each other.

An important parameter for describing fluids is density: $\rho=\frac{m}{V}$. Variations in density cause fluid parcels to rise or sink. For example, small variations in water density have a large impact on the circulation of water bodies because dense water tends to sink whereas less dense water tends to rise to the surface.

Why does dense material sink? To answer that, we need to think a bit about pressure.

[Switch to slides.]

Pressure is the force per area exerted by a fluid on a surface. We have already discussed this in the context of gases. It is a little bit different for liquids because liquids are weakly bonded to each other $\rightarrow$ discuss fluids in outerspace vs. on Earth $\rightarrow$ pressure arises due to temperature (especially for gases) and gravity (especially for liquids). But gas pressure is also affected by gravity, which is why the atmospheric pressure changes with altitude. Similarly, the pressure of liquids also depends on temperature because the density depends on temperature.


[Insert diagram with a parcel of water in a container.]
\vspace{5cm}

Let's assume that the parcel is in hydrostatic equilibrium.

There is a gravitational force acting on the parcel, a force from the overlying atmosphere that acts on the parcel of water, and a force that acts upward on the parcel due to the pressure from below. Summing the forces in the vertical direction,
$$\sum F_y = PA - F_g - P_oA = 0$$
$$PA - \rho Vg - P_0A = 0$$
$$PA - \rho Ahg - P_0A = 0$$
$$\boxed{P = \rho gh + P_0}$$
The hydrostatic fluid pressure depends on depth.

This pressure acts equally in all directions. If that wasn't the case, the fluid parcel would change shape. (One way to check would be to put a pressure sensor in water -- basically a device with a spring that responds to the force from the water -- and rotate it about in the water column.) This tells us that in a static fluid, water pressure is constant along horizontal lines and the water surface rises to the same level everywhere.

I always think its nice to have a reference value for physical units. A convenient one for pressure is to think about the water pressure below 10 m of water.
$$P=\rho gh+P_0=1000\mbox{ kg/m}^3\times 9.81\mbox{ m/s}^2\times 10\mbox{ m} + 10^5\mbox{ Pa} \approx 2\times 10^5\mbox{ Pa}$$
This is about twice the atmospheric pressure. In other words, 10 m of water is equivalent to several kilometers of air.

\subsubsection{Demo: Water shooting out of 2 L bottle}
Because water pressure increases with depth, the force acting on water molecules as they are pushed out of an opening will also depend on depth.

\subsection{Archimedes' principle}
Now let's look at the forces acting on an object in water. 

[Insert diagram of object submersed in water]
\vspace{5cm}

$$\sum F_y=F_{\mbox{below}}-F_{\mbox{above}}-F_g=F_{net}=ma_y$$
The forces from below and above the object are due to pressure (recall that $F=PA$).
$$\rho_w g(h+L)A-\rho_w ghA-\rho_o gLA=ma_y$$
$$\rho_w gLA-\rho_o gLA=\rho_oLAa_y$$
The upward force is referred to as the buoyant force. Notice that it is equal to the weight of the displaced fluid:
$$F_b=\rho_wgLA$$
This is referred to as ``Archimedes' Principle''. If the buoyant force exceeds the object's weight, the object will rise; if it is lower than the object's weight, the object will sink. 

Dividing by $LA$ and re-arranging, we arrive at
$$\frac{(\rho_w-\rho_o)}{\rho_o}g=a_y$$
An object that is more dense than water will sink, and the rate at which it sinks depends on the density differences. (If this we an object falling through air, $\rho_o>>\rho-w$ and so $a_y=-g$.0

\subsubsection{Demo: Cartesian diver}
Squeezing the bottle increases the water pressure, causing water to move into the diver and its density to increase.

\subsubsection{Demo: Weight in water}
Suspend aluminum weight from a scale. Note that when the reading on the scale drops when its placed in water.

With only a scale and water, can we determine the density of the hanging mass?

Its weight is
$$F_g = \rho Vg$$

What is its volume? When submerged in water,
$$\sum F_y = F_t-F_g+F_b = 0$$
$$F_t-F_g + \rho_w V g = 0$$
$$V=\frac{F_g - F_t}{\rho_w g}$$

Using the 100 g aluminum mass, $F_g = 1 N$ and $F_t = 0.65 N$.
$$V = 3.6\times 10^{-5}\mbox{ m}^3 = 36\mbox{ mL}$$
Plugging back into $F_g$,
$$\rho = \frac{F_g}{Vg}=2900\mbox{ kg m}^{-3}$$


\subsubsection{Example \#1: Buoy anchored to the ocean floor}
A sphere completely submerged in water is tethered to the bottom with a string. The tension in the string is one-third the weight of the sphere. What is the density of the sphere?

[Insert diagram.]
\vspace{5cm}

$$\sum F_y = F_b-F_g-F_t = ma_y = 0$$
$$F_t = \frac{1}{3}F_g$$
$$F_b-\frac{4}{3}F_g = 0$$
$$\rho_wgV-\frac{4}{3}\rho gV=0$$
$$\rho=\frac{3}{4}\rho_w=\boxed{750\mbox{ kg/m}^3}$$


\subsubsection{Icebergs}
I'm sure you've all heard that 90\% of an iceberg is below the water surface. We can explain that using Archimedes' principle.

We'll consider an iceberg floating at rest at the water surface, and ask how much of the iceberg sticks out of the water.

[Insert diagram.]
\vspace{5cm}

$$\sum F_y=F_b-F_g=0$$

$$F_b=m_dg=\rho_wV_dg,$$
where $V_d$ is the volume of displaced water.

$$F_g=m_ig=\rho_iV_ig$$

Combining these gives
$$\rho_wV_dg-\rho_iV_ig=0$$
Dividing by $g$ and re-arranging gives
$$\frac{V_d}{V_i}=\frac{\rho_i}{\rho_w}\approx 0.9$$
90\% of the iceberg is submerged.


\subsection{Measuring gas pressures}
Fluids are often used to measure gas pressure (such as atmospheric pressure)

\subsubsection{Manometer}
[Insert diagram.]
\vspace{5cm}

In a fluid that is in hydrostatic equilibrium, the pressure is constant along horizontal lines. (If this wasn't the case the fluid would move upward or downward...)

So in a manometer, this means that 
$$P_1=P_2$$
where
$$P_1=P_{gas}$$
and
$$P_2=P_o+\rho gh$$
Therefore, the gas pressure is
$$P_{gas}=P_o+\rho gh.$$

For this to work though, we need to know the atmospheric pressure. For that, we need a barometer.

\subsection{Barometer}
[Insert diagram of a barometer.]
\vspace{5cm}

In a barometer, 
$$P_o=\rho gh,$$
where $\rho$ is the fluid density and $h$ is the height of the column.

You could use any fluid, but its best to use really dense fluids (that way the barometer can be small). If you use water (with a density of 1000 kg/m$^3$), you would need a column that is
$$h=\frac{P_o}{\rho g}=10\mbox{ m}$$

Mercury has a density of 13,534 kg/m$^3$. The column of mercury that you need is only 0.76 m tall --- this is about 30 in of mercury.


 


\clearpage

