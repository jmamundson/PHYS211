%only use one spring? fit curve through peaks vs time to discuss exponential decay and calculate e-folding time? fit curve to find how oscillation frequency depends on mass and spring constant?

\documentclass[11pt,letterpaper]{article}
\usepackage{array}
\usepackage{fullpage}
\usepackage{graphicx}
\usepackage{parskip}
\usepackage{amsmath}
\usepackage[small]{caption}
\usepackage{graphpap}
\usepackage{logpap}
\usepackage{tabularx}
\usepackage{url}
\usepackage{hyperref}
\usepackage{enumitem}

\renewcommand{\thesection}{PART \arabic{section}: }

\newcounter{question}[section]
\newenvironment{question}[1][]{\refstepcounter{question}\par\medskip
   \textbf{\arabic{section}.\thequestion.} \rmfamily}{\medskip}

\usepackage{titlesec}
\titleformat{\section}{\clearpage\normalfont\bfseries}{\thesection}{0em}{}
\titlespacing{\section}{0pt}{0.5\baselineskip}{0pt}

\titleformat{\subsection}[runin]
{\normalfont\bfseries}{\thesubsection}{1em}{}

\titleformat{\subsubsection}{\normalfont\bfseries}{\thesubsubsection}{0em}{}
\titlespacing{\subsubsection}{0pt}{0.5\baselineskip}{0pt}

\newcounter{saveenumi}
\newcommand{\seti}{\setcounter{saveenumi}{\value{enumi}}}
\newcommand{\conti}{\setcounter{enumi}{\value{saveenumi}}}

\usepackage[dvipsnames]{xcolor}
\newcommand{\sol}[1]{{\color{NavyBlue} #1}}



\begin{document}
\setlength{\parindent}{0in}

%% EQUIP: Force Probe, Pulley, Track, Hangar, Weight, Tilt, Scale

\begin{flushright}
PHYS S211: General Physics I\\
Lab 8: Oscillations and Energy\\
11/8/22 (due 11/22/22)
\end{flushright}

Name:

\subsection*{Topics}
\begin{enumerate}
\setlength{\parskip}{3pt}
\item Hooke's Law
\item Simple harmonic motion
\item Energy associated with harmonic motion
\end{enumerate}

\subsection*{Introduction:}
Springs are used in many practical applications, ranging from closing screen doors to absorbing the shocks of highway bumps in automobile travel. Springs also serve as models for studying a number of complex physical phenomena including vibrations of atoms in a molecule, vibrations in crystals, motion of light, and the production and transmission of sound.  This lab consists of three sections that are designed to give you a better understanding of spring forces, simple harmonic motion (a type of oscillation), and the energy associated with motion of springs.

\subsection*{What you should turn in:} An individual formal lab report. See tips for writing reports and grading rubric on Blackboard under ``lab handouts''. % The lab report should consist of the following parts:
%\begin{itemize}
%\item Introduction: Discuss the topic/theory that you are exploring. Indicate in a brief statement or two how you will test that theory. (20 pts)
%\item Methods: Describe the experimental set-up. (15 pts)
%\item Results: Present data, including qualitative descriptions, and graphs. Draw the reader's attention to important aspects of the graphs. You should not discuss all of the minutiae of the data. (35 pts)
%\item Discussion and Conclusions: Summarize and interpret your results. Discuss whether or not your observations are consistent with theory, and explain why the differences might occur. Do not simply state that there was experimental error -- if there may have been experimental error, describe how it affected your calculations. Discuss how your assumptions may have affected your results. (30 pts)
%\end{itemize}

\subsection*{Equipment:}
\begin{itemize}
\setlength{\parskip}{3pt}  
\item Stand, pendulum holder, and tension spring
\item Mass hangar with several masses
\item LabQuest interface and motion sensor
\end{itemize}


\section{THE SPRING CONSTANT}
We have seen that the relationship between the force of a spring and the displacement away from the spring's equilibrium length is 
\begin{equation}
F_{sp}=-ky,
\label{eq:Hooke}
\end{equation} 
where $k$ is the spring constant and $y$ is the change in equilibrium position of the end of the spring. $F_{sp}$ is a restoring force --- it tries to restore the spring to its equilibrium length. Equation \ref{eq:Hooke} is known as Hooke's Law (though its not really a law) and is only valid for small displacements. In this part of the lab you will test Hooke's Law and compute the spring constant of a spring.

When a mass is added to a vertically hanging spring, and after the spring has stopped oscillating, 
\begin{equation}
\sum{F_y}=F_{sp}-F_g=0,
\label{eq:F_y}
\end{equation}
where $F_g=mg$ is the gravitational force acting on the hanging mass. In other words, there is no acceleration in the $y$-direction. As a result, 
\begin{equation}
F_s=F_g
\end{equation}
and so
\begin{equation}
y=-\frac{1}{k}F_g.
\label{eq:k}
\end{equation}
Equation \ref{eq:k} indicates that (1) Hooke's Law can be tested by hanging a variety of masses from a spring and seeing if $y$ ($y$-axis) vs. $F_g$ ($x$-axis) plots as a straight line and (2) the spring constant $k$ can be determined by finding the slope of that line. 

You will use the LoggerPro motion sensor to test Hooke's Law and to find the spring constant of a spring. Select a spring and several masses to suspend from the spring. Be sure to choose masses that give spring displacements that are relatively easy to measure.

Note that for the spring, ${y}$ is measured from the equilibrium point, so you zero the motion sensor for the equilibrium (un-stretched) length. To do this, hook the mass hangar over the spring but use your hand to keep the spring from stretching. Then press ctrl+0.

Now, suspend 6 different masses from the spring and use the motion sensor to determine ${y}$. Be careful that the mass does not fall on the motion detector! Record the spring displacements and associated masses. Plot ${y}$ vs. $F_g$ and use the slope of the best-fit line to find the spring constant, keeping in mind that the slope of the line is equal to $-1/k$ (Equation \ref{eq:k}). \textbf{Include this plot in your lab report and indicate the value of \textit{k}.}

%When finding the slope of the line, make sure that you set the $y$-intercept equal to 0 (Equation \ref{eq:k} does not have a $y$-intercept).

%(Note that if you had not zeroed the motion sensor, you still would have calculated the same value of $k$. However, the y-intercept in your plot would not equal 0.)

You will quite likely discover that the y-intercept of your line is not equal to zero (it should equal zero according to Hooke's Law). If this is the case, then the spring requires some non-zero force for it to start stretching. You will need to use the $y$-intercept when calculating the elastic potential energy in Section 3.

\section{OSCILLATION FREQUENCY}
You should try to empirically determine the relationship between the mass hanging on a spring and its oscillation frequency. Using five different masses, time how long it takes the spring to undergo 10 oscillations. You'll want to perform each measurement a few times and average the results to reduce the error in your measurements. \textbf{Submit a graph of frequency vs. mass.} Can you find a mathematical relationship between frequency and mass? (Hint: How does the graph look if you plot the data in log-log space?)

%Try performing the same measurements when the spring is hanging into a bucket of water. How (and why) do your results differ?

\section{OSCILLATIONS AND ENERGY\label{sec:energy}}
Using the same set-up as in Part 1, you will examine the relationship between the
potential energy, kinetic energy, and total energy of the spring
system. Place a mass on the spring, give the spring an initial displacement, and record the subsequent oscillations with the motion sensor. Record for about 1 min. Make sure that your initial displacement is not too large so that the spring does not oscillate wildly or become fully compressed. As in part 1, you should zero the motion sensor at the equilibrium (un-stretched) length. \textbf{Include plots of \textit{d} vs. \textit{t} and \textit{v} vs. \textit{t} in your lab report.} A system that oscillates due to a linear restoring force experiences sinusoidal variations in position, velocity, and acceleration. We refer to these types of motion as ``simple harmonic motion''. From your data, does it appear as though springs are harmonic oscillators?

Once the system begins to oscillate, there are no external forces acting on the system and so the energy of the system is conserved:
\begin{equation}
\Delta{E(t)}=0=\Delta{K(t)}+\Delta{U_g(t)}+\Delta{U_s(t)}+\Delta{E_{th}(t)},
\end{equation}
where $K$ is the kinetic energy, $U_g$ is the gravitational potential energy, $U_s$ is the elastic potential energy, and $E_{th}$ is the thermal energy. The kinetic and gravitational potential energies are given by 
\begin{equation}
\Delta{K(t)}=\frac{1}{2}m(v(t)^2-v_i^2).
\end{equation}
and 
\begin{equation}
\Delta{U_g(t)}=mg\Delta{y(t)}=mg(y(t)-y_i).
\end{equation}
Because the springs that you are using require some finite force before they start to stretch, you will need to use a slightly different equation for the elastic potential energy than we derived in class:
\begin{equation}
\Delta{U_s(t)}=\frac{1}{2}k(y(t)^2-y_i^2) - ky_0(y(t)-y_i)
\end{equation}
where $k$ and $y_0$ are the spring constant and y-intercept from part 1.

From the position and velocity data that you just collected, generate plots of $\Delta K(t)$, $\Delta U_g(t)$, and $\Delta U_s(t)$. \textbf{Include these plots in your lab report.} Discuss how these energy terms relate to each other. 

Add and plot $\Delta K(t)+\Delta U_g(t)+\Delta U_s(t)$. \textbf{Submit this graph with your report.} This graph should consist of a roughly straight line with no oscillations. This calculation is quite sensitive to the value of $k$ that you determined in part 1. If your value for $k$ is off, you will find that the graph oscillates in a sinusoidal fashion. If this is the case, try adjusting $k$ to make the line straighter. Is the sum of $\Delta K(t)+\Delta U_g(t)+\Delta U_s(t)$ constant? If not, does it become larger or smaller, and why? Compare the results from your two experiments.

\section{$e$-FOLDING TIME}
You should have observed that the oscillations decay over time in Part 3. The oscillations should decay approximately exponentially. Here, you should determine the e-folding time of the oscillations; the $e$-folding time is the time at which the amplitudes has decayed to $1/e$ times the initial amplitude. If an oscillation decays exponentially, then the amplitude $A$ varies with time:
\begin{equation}
A(t) = A_0e^{-t/\tau},
\end{equation}
where $A_0$ is the initial amplitude and $\tau$ is the $e$-folding time. When $t=\tau$, $A = A_0/e$. Exponential relationships can be transformed into linear relationships by taking the natural logarithm of both sides, such that
\begin{equation}
\ln A = \ln A_0 + \ln e^{-t/\tau} = \frac{-t}{\tau} + \ln A_0.
\end{equation}
Defining new variables $A^\prime = \ln A$ and $A_0^\prime = \ln A_0$, this becomes
\begin{equation}
A^\prime = -\frac{t}{\tau}+A_0^\prime,
\end{equation}
which is the equation for a straight line with a slope of $-1/\tau$. Your goal here is to determine $\tau$. This requires a couple of steps in MATLAB.

1. First, you must calculate the displacement of the mass relative to the mean position:
\begin{verbatim}
>> y = y - mean(y);
\end{verbatim}
\bigskip

2. Next determine the peaks.
\begin{verbatim}
>> [A, ind] = findpeaks(y);
\end{verbatim}
{\verb A } and {\verb ind } return the value of the peaks, or the amplitude, and the index, which can be used to find the time at which the spring has reached its maximum displacment during each oscillation.
\begin{verbatim}
>> t_peaks = t(ind);
\end{verbatim}
\bigskip

3. Transform the variables as done above, such that
\begin{verbatim}
>> A_prime = log(A);
\end{verbatim}
\bigskip

4. Now plot {\verb A_prime } vs. {\verb t_peaks } and find the slope of the best fit line. You can do this either by using the tool fitting feature in the graph or by using polyfit:
\begin{verbatim}
>> coeffs = polyfit(t_peaks, A_prime, 1)
\end{verbatim}
\bigskip

Note that your graph will most likely not be a perfectly straight line, which indicates that the decay is not exactly exponential. This occurs because the springs that we are using do not strictly obey Hooke's Law; if they did this should produce a straight line. Nonetheless exponential decay is a reasonable approximation of the decay that you observe.


\end{document}
