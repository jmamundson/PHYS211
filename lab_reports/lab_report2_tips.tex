\documentclass[11pt,letterpaper]{article}
\usepackage{array}
\usepackage[top=1in,bottom=1in,right=1in,left=1in]{geometry}
\usepackage{graphicx}
\usepackage{parskip}
\usepackage{amsmath}
\usepackage[small]{caption}
\usepackage{graphpap}
\usepackage{tabularx}

\begin{document}
Tips for writing lab reports:

Lab reports serve several purposes

1. Force you to learn concepts through critical thinking: perform experiments, think about experiments, and communicate results. \\
2. Demonstrate your participation in the lab.\\
3. Practice technical writing.\\

There are three questions to keep in mind when writing lab reports.

1. Could your friend (or your boss) read the report and understand exactly what you did?\\
2. Could you read the report a month later and repeat the experiment and get the same results?\\
3. Could another scientist read the abstract/introduction (see comments below on abstracts) and immediately decide whether to take the time to read the full report?

%The reports for some labs will be fairly informal - a few paragraphs and plots describing the experiments that you performed in lab. For these labs I will focus on whether or not you have learned the concepts.

%Other labs will require a full formal lab report. These reports are worth more. The first formal lab will consist of one revision...

Things to include in the formal reports:\\
0. Abstract\\
1. Introduction\\
2. Methods (including materials and procedures)\\
3. Results (including observations, calculation, data, and figures)\\
4. Discussion \& Conclusions (including answers to specific questions asked within the lab)\\
   
Details follow:\\

\textbf{Abstract:} Due to the short nature of these labs, I find the abstract to be a bit unnecessary and so do not require them. However, a report would typically begin with an abstract, which summarizes in a paragraph or two (at most) the purpose of the experiment, as well as the key results and significance and major conclusion(s) you drew from these results.  Any numerical results should be expressed with the proper number of significant digits, matching the uncertainties you identified, e.g. ``We found the acceleration of gravity to be $g=9.71\mbox{ m/s}^2 \pm 0.25\mbox{ m/s}^2$.''\\

 
\textbf{Introduction:} Describes the experiment, and why it was important. This may require a series of paragraphs where you describe the scientific principles you were modeling and give background historical and theoretical information. In longer and more complex experiments, you would include results of prior relevant research, too. 

The introduction should end with a purpose statement-– one sentence that specifically states the question(s) the experiment was designed to answer, e.g. ``The purpose of this investigation was to determine the effects of gravity on a pendulum swinging in the presence of air resistance.''  For some experiments, it may be more appropriate to state the hypothesis that was being tested, e.g. ``The hypothesis was that current moving in a conductive material will generate a magnetic field around that material.''\\

 
\textbf{Methods:} Describes in narrative form how you did the experiment, including information about the equipment you used, the steps you took, and the methods of gathering and analyzing the data. This section must include complete details and be written clearly to allow readers to duplicate the experiment if they wish. Your goal for this part of the report is to allow anyone reading the report to duplicate your results. Good science demands that results that are to be trusted must be independently verified by other observers. This section is not a list of materials and list of instructions copied from a laboratory manual.\\

 
\textbf{Results:} Summarizes your observations, data, calculations, and results.  (e.g. ``The value of $g$ was found to $8.9\mbox{ m/s}^2 \pm 0.2\mbox{ m/s}^2$,''  or, ``The thermal conductivity of copper obtained from the experiment was $386\mbox{ W/(M}\cdot\mbox{K)}$.'' Note that you should include in your results an estimate of the uncertainty of your calculations and experiment.

Raw data is sometimes most effective in a graph, unless the experiment only consisted of a few trials --- in which case you might use a table. Data should be recorded in a neat and orderly fashion, either on data tables provided in a lab handout, or in tables you create for the activity. Data must include the units of measurement and those should appear consistently throughout calculations. Sample calculations illustrating how you derived your results can be included here, including any statistical tests or analysis applied. But, leave the interpretation of your data and interpretation of any statistical test applied to the data for the Discussion and Conclusions section. 

Remember that no experiment is 100\% precise, because all measurements – made by humans or electronic data gathering tools – are imprecise to some limit.  Be sure to reasonably estimate what your uncertainties in measurements are (and record those in your data tables). You can then account for the impact these measurement uncertainties have upon your results in the next section. 

For example, if you were measuring the conductivity of a metal bar, and measured the diameter as $5.0\mbox{ cm}\pm 0.1\mbox{ cm}$ (in other words, to within a millimeter), your measurement is at least 1/50 = 2\% uncertain.  It could really have been between 4.9 cm and 5.1cm, and this uncertainty will translate to an uncertainty in your overall derived or calculated value.

Graphs and figures will appear here, and they must be clearly labeled, with any slopes or intercepts or regression lines and coefficients of determination included in the graph (generated by software tools.) Any additional actual data pages from the experiment (including data written on pre-printed handout labs passed out to the class, or copies from your lab notebook) can go in the appendix.

Please Note: It is ESSENTIAL that you accurately report your findings and not manufacture data to fit preconceptions. ``The greatest advances in science occur when a careful researcher gets reproducible data which conflict with accepted notions.'' You will be graded on how you analyze and interpret your data, not on whether it agrees with your expectations.\\


\textbf{Discussion and Conclusions:} Here I look for evidence that you have critically analyzed your data. This is where you impress your readers with your analysis of the data and your interpretation of your results, including a comparison with known values if appropriate, an evaluation of any statistical tests you used (mean, standard deviation, least-squares fit, coefficient of determination, etc.) What did your results mean? Did your results confirm or conflict with your initial expectations? Was your hypothesis supported or contradicted, or were the results not sufficiently precise to allow you to say one way or the other? 
 
If your results differed from known values, how do you account for the error?  Did it arise from measurement uncertainties, the procedure, the equipment, or a combination?  If so, how did these factors influence your results?  Be as specific as possible here – not just mentioning sources of error, but \textit{analyzing them for their expected impact on the results}. Interpretations should be supported whenever possible by references to the lab handout, your text, and/or other studies from the literature that can be properly cited. 

Analyze your results, looking for trends or patterns.  Were you consistently above expected values, or below?  Were your errors random?  Did your results differ from other groups in the class doing the same experiment? Propose hypotheses to account for any differences or patterns you identified.

If you graphed data elements, and analyzed those plots, discuss what the graphs revealed.  Do not merely restate your data section; rather, make generalizations --- e.g., ``As the initial velocity is doubled, the stopping distance increases by a factor of approximately four.'' Since you have plotted data, state the equation which describes the data and discuss its significance.

If your lab handout includes general questions to be answered, include those answers here.

Good lab reports read forward as well as backward, and the conclusion of your discussion (and your report) might well include predictions of how the experiment might have produced different results if done differently, with better equipment, or different procedures. Do not spend enormous amounts of time explaining data that cannot be explained. Most importantly, you should offer ``supported'' statements or what was determined from doing the experiment. In other words, tell me what your conclusions were with supporting evidence confirming your beliefs.

 
\end{document}

 
 

 

%Tips and Hints for Effective Physics Laboratory Reports

 

 

%The Purpose of the Report is to Communicate Truth. More than just data and results, you want your report to communicate the “truth” about the entire laboratory experiment or activity – by making and recording real measurements, reading as precisely as possible but not assumed to be more precise than possible, and then drawing conclusions by analyzing actual data, with all uncertainties and errors that might arise.  You want your readers to be able to quickly grasp what experiment you did, why it was important, how you did it, why, what results you obtained, and what you think the significance of those results might be.  Making the data appear to be better than was is not good science. 

%Accuracy is important. A Science Lab Report is not creative writing. Stay within the limits of your study or experiment. If your report says something you did not intend, then either you did not understand what you were doing, or your reader will not know what you were doing (or both).

 

%Clarity is important. Do not assume that science (or an essay topic) is too complicated for you to understand. If you do not understand what you have written, then no one else will understand your report either. If you have filled your report with scientific terms and complicated sentences in order to sound sophisticated, then you have probably succeeded only in sounding muddled.

 

%Grammar and spelling are important.  If you use computer grammar checkers, be aware that they are probably not yet able to handle scientific style.  Use a spell checking tool, but be aware that most spelling checkers do not include many scientific terms. Copy unfamiliar terms with great care. A few letters can make a big difference in a scientific term.

 

%Common Errors and Pitfalls to Avoid:

 

%    Failing to follow formal lab report guidelines, or to label the sections of the lab report such that it is not clear whether a given part of it is reporting on the procedure, the data or the conclusions.
%    Placing procedural statements in the Abstract or Introduction, e.g., "we will then measure the photogate time at point A and calculate the speed, repeating the measurements three times to insure accuracy.") Or similarly, to place procedural statements in the Discussion section (e.g., "we measured the photogate time and calculated the instantaneous speed and then found the kinetic energy and total mechanical energy.").  These should be in the experimental procedures section.
%    Neglecting to include a solid theoretical background, or merely stating an obvious, short hypothesis (e.g., "we believe that the higher the initial height, the greater the speed at point B."). The theory should be sufficient to demonstrate that you know how the laws of physics applied.
%    Neglecting to include a detailed procedure which would provide sufficient direction for anyone to follow. The steps should be included in a narrative form where there is enough detail for a person who is unfamiliar with the equipment to conduct the same study.
%    Using the first person in the procedure section, e.g. "I then placed selected the rectangle tool. Then I drew a box by dragging on the screen. I then ...".   In a group experiment, “we” should be used.
%    Restating your measured data in the Discussion and Conclusion section (e.g., "we measured the photogate times to be 0.0125 s when the height was 0.5 m and we measured the time to be 0.008 s when the height was 0.15 m").
%    Making very general conclusions which (while perhaps true) have nothing to do with the idea behind the lab report (e.g., "This project was fun and we learned a lot. We hope that we can do more projects like this.").
\end{document}