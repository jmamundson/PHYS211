\documentclass[11pt,letterpaper]{article}
\usepackage[top=1in,bottom=1in,left=1in,right=1in]{geometry}
\pagestyle{empty}
\renewcommand{\labelitemi}{$\cdot$}

\begin{document}
\noindent Basic definitions of kinematics
\begin{itemize}
\item position: $\vec{x}=\langle{x,y}\rangle$
\item displacement: $\Delta{\vec{x}}=\vec{x}_f-\vec{x}_i$ for finite displacement
\item instantaneous velocity: $\vec{v}=\displaystyle\frac{d\vec{x}}{dt}$
\item instantaneous acceleration: $\vec{a}=\displaystyle\frac{d\vec{v}}{dt}=\displaystyle\frac{d^2\vec{x}}{dt^2}$
\end{itemize}
Kinematic equations, $\vec{v}=\mbox{constant}$ (i.e., $\vec{a}=0$)
\begin{itemize}
\item $\vec{v}=\displaystyle\frac{\Delta{\vec{x}}}{\Delta{t}}$
\item $\vec{x}_f=\vec{x}_i+\vec{v}\Delta{t}$
\end{itemize}
Kinematic equations, $\vec{a}\neq{0}$ and $\vec{a}=\mbox{constant}$
\begin{itemize}
\item $\vec{a}=\displaystyle\frac{\Delta{\vec{v}}}{\Delta{t}}$
\item $\vec{v}_f=\vec{v}_i+\vec{a}\Delta{t}$
\item $\Delta{\vec{x}}=\vec{v}_i\Delta{t}+\displaystyle\frac{1}{2}\vec{a}\Delta{t}^2$
\item $\left(v_{x,f}\right)^2-\left(v_{x,i}\right)^2=2a_x\Delta{x}$ and $\left(v_{y,f}\right)^2-\left(v_{y,i}\right)^2=2a_y\Delta{y}$
\item common example of $a=\mbox{constant}$ is $g=9.81\mbox{ m/s}^2$
\end{itemize}
Motion of object A relative to object C
\begin{itemize}
\item $\vec{v}_{ac}=\vec{v}_{ab}+\vec{v}_{bc}$
\end{itemize}

\noindent Basic definitions for circular and rotational motion
\begin{itemize}
\item angular position: $\theta\mbox{(radians)}=\displaystyle\frac{s}{r}$; $s=$arclength, $r=$radius
\item angular displacement: $\Delta\theta=\theta_f-\theta_i$
\item angular velocity: $\omega=\displaystyle\frac{d\theta}{dt}=2\pi{f}=\displaystyle\frac{2\pi}{T}$; $f=$frequency, $T=$period
\item angular acceleration: $\alpha=\displaystyle\frac{d\omega}{d{t}}=\frac{d^2\theta}{dt^2}$
\end{itemize}
Kinematic equations for constant angular acceleration
\begin{itemize}
\item $\omega_f=\omega_i+\alpha\Delta{t}$
\item $\Delta\theta=\omega_i\Delta{t}+\displaystyle\frac{1}{2}\alpha\Delta{t}^2$; if $\alpha=0$, $\omega_f=\omega_i=\mbox{constant}$.
\item $\omega_f^2-\omega_i^2=2\alpha\Delta{\theta}$
\end{itemize}

\clearpage
\noindent Speed, acceleration, and forces
\begin{itemize}
\item speed: $v=\omega{r}$
\item centripetal acceleration: $a_c=\displaystyle\frac{v^2}{r}=\omega^2r$
\item centripetal force: $F_c=ma_c=m\displaystyle\frac{v^2}{r}=m\omega^2r$; points toward center of circle
\item tangential acceleration: $a_t=\alpha{r}$
\end{itemize}
Newton's Laws
\begin{enumerate}
\item if $\vec{F}_{net}=0\Rightarrow \vec{a}=0$
\item $\displaystyle\sum\vec{F}=m\vec{a}$
\item $\vec{F}_{12}=-\vec{F}_{21}$
\end{enumerate}
Types of forces
\begin{itemize}
\item Newton's Law of Gravity: $F_{12}=F_{21}=\displaystyle\frac{Gm_1m_2}{r^2}$; points from one object to another; this can also be expressed as $\vec{F}_{12}=\displaystyle-\frac{Gm_1m_2}{r^2}\hat{r}_{12}$ where $\hat{r}_{12}$ is the unit vector that points from object 1 to object 2.
\item Gravitational constant: $G=6.67\times{10}^{-11}\mbox{ N}\cdot\mbox{m}^2/\mbox{kg}^2$
\item For objects near the Earth's surface, $F_g=mg$
\item normal force, $\vec{F}_n$, is perpendicular to surface and prevents objects from penetrating the surface
\item frictional force depends on $\vec{F}_n$ and $\vec{v}$
\begin{itemize}
\item if $\vec{v}=0$ use static friction; $\vec{F}_s$ balances other forces as long as $\left|\vec{F}_s\right|\leq\mu_s\left|\vec{F}_n\right|$
\item if $\vec{v}\neq{0}$ use kinetic friction; $\left|\vec{F}_k\right|=\mu_k\left|\vec{F}_n\right|$
\end{itemize}
\item tensional force, $\vec{F}_t$, is transmitted through a rope and around pulleys
\end{itemize}
Impulse and momentum
\begin{itemize}
\item momentum, $\vec{p}=m\vec{v}$
\item Newton's second law, $\displaystyle\sum\vec{F}=\displaystyle\frac{d\vec{p}}{dt}$
\item Impulse-Momentum Theorem, $\vec{J}=\Delta{\vec{p}}=\displaystyle\int_{t_0}^{t_1}\vec{F}_{net}\,dt=\vec{F}_{avg}\Delta{t}$
\item total momentum, $\vec{P}=\vec{p}_1+\vec{p}_2+...$
\item conservation of momentum, $\Delta{\vec P}=0$ if $\vec{F}_{net}=0$ or if $\vec{F}_{net}$ is small compared to other forces for short $\Delta{t}$
\end{itemize}

Orbits and gravity
\begin{itemize}
\item orbital velocity: $v=\displaystyle\sqrt{\displaystyle\frac{GM}{r}}$ at a distance of $r$ from the center of a planet with mass $M$
\end{itemize}
Torque and moment of inertia
\begin{itemize}
\item Torque: $\tau=rF_\perp$; $F_\perp$ is force perpendicular to radial axis; depending on what angles are given, you may find that $F_\perp=F\sin\phi$
\item Newton's Second Law for rotation: $\sum\tau=I\alpha=\displaystyle\frac{dL}{dt}$; $L=I\omega=$ angular momentum.
\item Moment of inertia: $I$, indicates how difficult it is to rotate an object
\item Conservation of angular momentum: if $\tau_{ext}=0$ then $\Delta{L}=0$.
\item Rolling constraint: $v=\omega{R}$ and $a=\alpha{R}$; rolling object, with perfect friction, moves forward with this velocity
\end{itemize}
Static equilibrium
\begin{itemize}
\item $\sum{F_x}=0$; $\sum{F_y}=0$; $\sum\tau=0$
\item Choose convenient pivot point
\end{itemize}
Stability and balance
\begin{itemize}
\item Does torque restore object to its original position?
\item Critical angle: $\theta_c=\tan^{-1}\left(\displaystyle\frac{t}{2h}\right)$; $t=$width of object's base, $h$ is the height to the center of mass
\end{itemize}
Springs and elasticity
 \begin{itemize}
 \item Hooke's `Law': $F=-k\Delta{x}$; $k$ is empirically determined spring constant
 %\item For elastic materials, $k=\displaystyle\frac{YA}{L}$; $Y=$Young's modulus
% \item Hooke's Law sometimes written $\displaystyle\left(\frac{F}{A}\right)=Y\left(\displaystyle\frac{\Delta{L}}{L}\right)$; $\mbox{stress}=Y\times\mbox{strain}$
\end{itemize}
Energy and work
 \begin{itemize}
 \item Total energy: $E=K+U_g+U_s+E_{th}+...$
 %\item Work: $W=\displaystyle\int_{x_0}^{x_1}F_{ext}\,dx=\Delta{E}$; mechanical transfer of energy into or out of a system
 \item Conservation of energy: for an isolated system, $F_{ext}=0$ and therefore and therefore $W=\Delta{E}=0$.
 \item Work: $W=F_\parallel\cdot{d}$; force $\times$ displacement
 \item Translational kinetic energy: $K_{trans}=\displaystyle\frac{1}{2}mv^2$; scalar quantity (note that $v$ is speed, not velocity)
 \item Rotational kinetic energy: $K_{rot}=\displaystyle\frac{1}{2}I\omega^2$
 \item Total kinetic energy: $K=K_{trans}+K_{rot}$
 \item Gravitational potential energy: $\Delta{U_g}=mg\Delta{y}$; choose convenient reference height
 \item Elastic potential energy: $U_s=\displaystyle\frac{1}{2}k(\Delta{x})^2$; $k$ is the spring constant
 \item Thermal energy (from friction): $\Delta{E_{th}}=F_f\Delta{x}$
 \item Thermal energy (from collision in which one object is initially stationary):\\
 $\Delta E_{th} = K_i\left(\frac{m_2}{m_1+m_2}\right)\left(1-C_r^2\right)$, where $C_r$ is the coefficient of restitution for the collision and is given by $C_r = (v_{2,f}-v_{1,f})/v_{1,i}$
 \item Conservation of mechanical energy: for isolated system with no friction, $\Delta{K}+\Delta{U_g}+\Delta{U_s}=0$
 \item Power: $P=\displaystyle\frac{d{E}}{d{t}}$
 \end{itemize}


%% Thermodynamics
%% \begin{itemize}
%% \item 1st Law: $Q+W=\Delta{E}$; $Q$ is heat transferred into or out of the system
%% \item 2nd Law: Entropy in a closed system (which describes disorder) can never decrease
%% \item energy needed to change material's temperature: $Q=mc\Delta{T}$; $c$ is the specific heat capacity, a material property
%% \item energy need to change a material's phase: $Q=\pm mL_f$ to melt solid and $Q=\pm mL_v$ to turn liquid into gas; use negative sign if going from gas to liquid or liquid to solid
%% \item $L_v>L_f$ (specific latent heat of vaporization is greater than specific latent heat of fusion)
%% \item conduction: $\frac{Q}{\Delta{t}}=\left(\frac{kA}{L}\right)\Delta{T}$; $k$ is thermal conductivity
%% \item advection/convection: need to solve for motion of an object
%% \item radiation: $\frac{Q}{\Delta{t}}=e\sigma{A}T^4$; $e$ is emissivity and $\sigma=5.67\times{10}^{-8}\frac{\mbox{W}}{\mbox{m}^2\mbox{K}^4}$ is the Stefan-Boltzmann constant
%% \item ideal gas law: $PV=nRT$, where $R=8.314\mbox{ J/(mol}\cdot\mbox{K)}$ is the gas constant
%% \item for ideal gas in enclosed, insulated container, $\frac{PV}{T}=\mbox{constant}$
%% \item work done by expanding a gas at constant pressure: $W_{\mathrm{gas}}=P\Delta{V}$
%% \item note that for gases, specific heat at constant pressure $c_p$ differs from specific heat at constant volume $c_v$. $c_p$ is approximately 50$\%$ larger than $c_v$
%% \item Conversions: 1 cm$^3$=10$^{-6}$ m$^3$; 1 atm = 101.3 kPa
%% \end{itemize}
%% Fluids
%% \begin{itemize}
%% \item density: $\rho=\frac{m}{V}$
%% \item hydrostatic pressure: $P=\rho{g}d+P_0$; $P_0$ is pressure from above
%% \item gauge pressure: $P_g=P_{gas}-P_{atm}$; $P_{atm}=101.3\mbox{ kPa}$
%% \item Archimedes' principle (upward buoyant force): $F_b=\rho_f{g}V$; $\rho_f$ is fluid density and $V$ is volume displaced
%% \item flux: $Q=vA$; $Q$ is constant in a pipe for ideal fluid
%% \item Bernoulli's equation: along a streamline, $P+\frac{1}{2}\rho{v^2}+\rho{g}h=\mathrm{constant}$
%% \end{itemize}

%% Simple harmonic motion
%% \begin{itemize}
%% \item displacement: $x(t)=A\cos(2\pi{f}t)$
%% \item velocity: $v(t)=-v_{\mathrm{max}}\sin(2\pi{f}t)$
%% \item acceleration: $a(t)=-a_{\mathrm{max}}\cos(2\pi{f}t)$
%% \item $v_{\mathrm{max}}=2\pi{f}A$ and $a_{\mathrm{max}}=(2\pi{f})^2A$
%% \item for a spring, $f=\frac{1}{2\pi}\sqrt{\frac{k}{m}}$; for a pendulum, $f=\frac{1}{2\pi}\sqrt{\frac{g}{L}}$
%% \item damped oscillation: $x(t)=A\exp^{-t/\tau}\cos(2\pi{f}t)$, where $\tau$ is a decay constant
%% \end{itemize}
%% Waves
%% \begin{itemize}
%% \item in general, $v=\lambda/T$, where $\lambda$ is wavelength and $T$ is period
%% \item velocity of a wave on a string: $v=\sqrt{\frac{F_t}{\mu}}$, where $\mu$ is linear density
%% \item velocity of sound waves in ideal gas: $v=\sqrt{\frac{\gamma{RT}}{M}}$, where $\gamma$ is adiabatic index, $R$ is the gas constant, $T$ is temperature in Kelvin, and $M$ is the molar mass
%% \item travelling wave displacement: $y(x,t)=A\sin\left(2\pi\frac{x}{\lambda}\pm 2\pi\frac{t}{T}+\phi\right)$; Use $-$ if wave is travelling to the right, $+$ if it is travelling to the left
%% \item Doppler effect: $f=f_0\displaystyle\left(\frac{v}{v\pm{v_s}}\right)$; $v_s$ is speed of source, $v$ is wave speed. Use $+$ if the source is moving away from the receiver, $-$ if the source is moving toward the receiver.
%% \item wave superposition: $y(x,t)=y_1+y_2+y_3$ where $y_i$ are given by different amplitudes, wavelengths, and frequencies/periods.
 %% \item standing wave displacement: $y(x,t)=A\sin\left(2\pi\left(\frac{x}{\lambda}-\frac{t}{T}\right)\right)+A\sin\left(2\pi\left(\frac{x}{\lambda}+\frac{t}{T}\right)\right)$
%% \item harmonics: $\lambda_m=\frac{2L}{m}$, where $m=1,2,3$\ldots; for a string that is pinned on both ends, a tube that is open on both ends, and a tube that is closed on both ends.
%% \item harmonics: $\lambda_m=\frac{4L}{m}$, where $m=1,3,5$\ldots; for a tube that is open on one end and closed on the other
%% \end{itemize}

\end{document}
